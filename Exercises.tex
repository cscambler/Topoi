\documentclass{article}

\usepackage[margin=1.5in]{geometry} 
\usepackage{amsmath,amsthm,amssymb,hyperref, tikzcd, stmaryrd}


\newcommand{\Z}{\mathbf{Z}}
\newcommand{\N}{\mathbf{N}}
\newcommand{\Q}{\mathbf{Q}}
\newcommand{\lb}{\llbracket}
\newcommand{\rb}{\rrbracket}

\newenvironment{theorem}[2][Theorem]{\begin{trivlist}
\item[\hskip \labelsep {\bfseries #1}\hskip \labelsep {\bfseries #2.}]}{\end{trivlist}}
\newenvironment{lemma}[2][Lemma]{\begin{trivlist}
\item[\hskip \labelsep {\bfseries #1}\hskip \labelsep {\bfseries #2.}]}{\end{trivlist}}
\newenvironment{claim}[2][Claim]{\begin{trivlist}
\item[\hskip \labelsep {\bfseries #1}\hskip \labelsep {\bfseries #2.}]}{\end{trivlist}}
\newenvironment{ex}[2][Exercise]{\begin{trivlist}
\item[\hskip \labelsep {\bfseries #1}\hskip \labelsep {\bfseries #2.}]}{\end{trivlist}} 
\newenvironment{problem}[2][Problem]{\begin{trivlist}
\item[\hskip \labelsep {\bfseries #1}\hskip \labelsep {\bfseries #2.}]}{\end{trivlist}}
\newenvironment{proposition}[2][Proposition]{\begin{trivlist}
\item[\hskip \labelsep {\bfseries #1}\hskip \labelsep {\bfseries #2.}]}{\end{trivlist}}
\newenvironment{corollary}[2][Corollary]{\begin{trivlist}
\item[\hskip \labelsep {\bfseries #1}\hskip \labelsep {\bfseries #2.}]}{\end{trivlist}}

\newenvironment{solution}{\begin{proof}[Solution]}{\end{proof}}

\begin{document}

\large 
\linespread{1.5} 

{\Large Topoi 
\hfill  CJS}

\begin{ex}[Monic arrows in $\Omega$-set]
    . 
    An arrow $f : A \to B$ in $\Omega$-set is monic iff it satisfies
    \begin{equation}\label{hyp}
        f(a_0, b) \wedge f(a_1, b) \leq \lb a_0 = a_1\rb_A
    \end{equation}
\end{ex}
\begin{proof}
    Let $g : C \to A$ in $\Omega$-set. 
    
    We first prove that (1) implies
    \begin{equation}\label{eq}
        g(c, a_0) = \bigsqcup_{b \in B} g\circ f(c, b)\wedge f(a_0, b)
    \end{equation}
    for any $c, a_0$. To do this we show eacch of 
    \begin{equation}\label{leq}
        g(c, a_0) \leq \bigsqcup_{b \in B} g\circ f(c, b)\wedge f(a_0, b)
    \end{equation}
    and 
    \begin{equation}\label{geq}
        g(c, a_0) \geq \bigsqcup_{b \in B} g\circ f(c, b)\wedge f(a_0, b)
    \end{equation}
    For \eqref{leq}, first observe
    \begin{equation}\label{w}
        g(c, a_0) \leq \lb a_0 = a_0 \rb \leq \bigsqcup_{b\in B} f(a_0, b)
    \end{equation}
    we can therefore chose $b \in B$ with 
    \begin{equation}\label{wv}
        g(c, a_0) \leq \lb a_0 = a_0 \rb \leq f(a_0, b)
    \end{equation}
    \eqref{wv} implies $g(c, a_0) \leq g(c, a_0) \wedge f(a_0, b)$
    which in turn implies 
    \begin{equation}
        g(c, a_0) \leq (\bigsqcup_{a \in A} g(c, a) \wedge f(a, b)) \wedge f(a_0, b)
    \end{equation}
    from which \eqref{leq} follows trivially.

    For \eqref{geq}, let $b \in B$. We have:
    \begin{equation}\label{t1}
        (\bigsqcup_{a \in A} g(c, a) \wedge f(a, b)) \wedge f(a_0, b) = \bigsqcup_{a \in A} g(c, a) \wedge f(a, b) \wedge f(a_0, b)
    \end{equation}
    \begin{equation}\label{t2}
        \leq \bigsqcup_{a \in A} g(c, a) \wedge \lb a = a_0 \rb_A
    \end{equation}
    \begin{equation}\label{t3}
        \leq g(c, a_0)
    \end{equation}
    where the transition from \eqref{t1} to \eqref{t2} uses \eqref{hyp},
    and the transition from \eqref{t2} to \eqref{t3} uses (v) p 277.
    (Is equality true?)

    From here it is easy. Suppose $g\circ f(c, b) = h\circ f(c, b)$ for all
    $c$ and $b$. Then for any $a$, $\bigsqcup_{b \in B} g\circ f(c, b) \wedge f(a, b) = \bigsqcup_{b \in B} h\circ f(c, b) \wedge f(a, b)$.
    Hence $g(c, a) = f(c, a)$ by (2).
\end{proof}

\end{document}